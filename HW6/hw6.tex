\documentclass[12pt,letterpaper]{article}

\usepackage{graphicx,amssymb,amsmath,bm,color,multicol}
\usepackage{../newcommand}
\sloppy
\newcommand{\ignore}[1]{}

\newenvironment{proof}{\noindent{\bf Proof:}}{\qed\bigskip}

\newtheorem{theorem}{Theorem}
\newtheorem{corollary}{Corollary}
\newtheorem{lemma}{Lemma} 
\newtheorem{claim}{Claim}
\newtheorem{fact}{Fact}
\newtheorem{definition}{Definition}
\newtheorem{assumption}{Assumption}
\newtheorem{observation}{Observation}
\newtheorem{example}{Example}
\newcommand{\qed}{\rule{7pt}{7pt}}

\newcommand{\homework}[4]{
	\thispagestyle{plain} 
	\newpage
	\setcounter{page}{1}
	\noindent
	\begin{center}
		\framebox{ \vbox{ \hbox to 6.28in
				{\bf ECON 4190: Industrial Organization \hfill #1} %change course name
				\vspace{4mm}
				\hbox to 6.28in
				{\hspace{2.5in}\large\mbox{Homework #2}}
				\vspace{4mm}
				\hbox to 6.28in
				{{\it Collaborators: #3\hfill}}
			}}
		\end{center}
	}

\oddsidemargin 0in
\evensidemargin 0in
\textwidth 6.5in
\topmargin -0.5in
\textheight 9.0in

\begin{document}

% Modify this command to be your name and computing ID
\homework{Fall 2021}{$6$}{Alex Shen (as5gd), Sean Velhagen (spv5hq), Max Bresticker (mtb9sex)}

Pledge: On my honor, I pledge that I have neither given nor received help on this assignment 
Signature: \textit{Alex Shen, Sean Velhagen, Max Bresticker}

\begin{enumerate}
	
\item[1.] 

\begin{enumerate}
	\item Firms on the competitive fringe are price-takers, so they take $P$ as given and produce a quantity where $MC=P$. In this case, $P = MC = 10 + 50q_i$, so an individual fringe firm will produce $q_i = \frac{1}{50}P - \frac{1}{10}$. Since there are 100 firms, the total supply for the competitive fringe $Q_f = 100 q_i = 2P - 20$. Note that this only holds if $P>10$, as otherwise this model implies that firms would produce a negative quantity (when in reality they would shut down and produce nothing).
	
	{\color{blue}\textbf{Solution:} The supply curve for the competitive fringe is $Q_f = 2P-20$ where $P > 10$, otherwise 0.}
	\item To solve for residual demand, all we need to do is take market demand $Q_m = 400 - 8P$ and subtract $Q_f$ from it. Since the residual demand is the demand for the dominant firm, we will call it $Q_d = 400 - 8P - (2P - 20) = 420 - 10P$.
	{\color{blue}\textbf{Solution:} Residual demand faced by the dominant firm is $Q_d = 420 - 10P$}
	\item Given the residual demand, we can simply solve this like any other monopolist problem, where the objective is to maximize the dominant firm's profit $\pi_d$ by setting price:
	\begin{align*}
		max p \pi_d &= (420-10P)(P -10) \\
		\frac{\partial\pi}{\partial p} &= (420-10P) + (-10)(P - 10) \\
		0 &= 420-10P-10P+100 \\
		P &= 520 / 20 = 26
	\end{align*}
	Now that we know $P$, we can solve for everything else quite trivially; $Q_d = 420 - 10(26) = 160, Q_f = 2(26) - 20 = 32$, and the market shares for the dominant firm and the fringe are $\frac{160}{160 + 32} = \frac{5}{6}$ and $\frac{1}{6}$ respectively.

	{\color{blue}\textbf{Solution:} $P = 26, Q_d = 160, Q_f = 32$, and market shares for dominant firm and fringe are $\frac{5}{6}$ and $\frac{1}{6}$ respectively.}
	\item Recall that $Q_d = 420 - 10P$, or equivalently, $P = 42 - \frac{1}{10}Q$. Considering that now $Q$ is equal to $q_1 + q_2$ (the combined total that firms 1 and 2 produce), we can solve for both firms' profit-maximizing quantity via symmetry:
	\begin{align*}
		max_{q_1} \pi_1 &= (P - c)(q_1) \\
		&= (42 - \frac{1}{10}q_1 - \frac{1}{10}q_2 - 10)(q_1) \\
		\frac{\partial \pi_1}{\partial q_1} &= 32 - \frac{2}{10}q_1 - \frac{1}{10}q_2 = 0 \\
		q_1 &= \frac{10}{2} (32 - \frac{1}{10}q_2) \\
		&= 160 - \frac{1}{2}q_2 \\
		\therefore q_2 &= 160 - \frac{1}{2}q_1 \\
		\therefore q_1 &= 160 - \frac{1}{2}(160 - \frac{1}{2}q_1) \\
		&= 80 + \frac{1}{4}q_1 \\
		q_1 = q_2 &= \frac{4}{3} *80 = \frac{320}{3} \\
		\therefore Q_d &= \frac{640}{3} \approx 213.33
	\end{align*}

	Plugging into the inverse residual demand function, we get $P = 42 - \frac{1}{10} * \frac{640}{3} = \frac{126 - 64}{3} = \frac{62}{3} \approx 20.667$. Recalling the fringe demand $Q_f = 2P-20$, we can see that $Q_f = \frac{124}{3} - \frac{60}{3} = \frac{64}{3} \approx 21.333$. Fringe market share is $\frac{64}{3} / \frac{64 + 640}{3} = \frac{1}{11}$.


	{\color{blue}\textbf{Solution:} In Cournot, $q_1 = q_2 = \frac{320}{3} \therefore Q_d = \frac{640}{3} \approx 213.333$ setting market price to $P=\frac{62}{3} \approx 20.667$, fringe produces $Q_f = \frac{64}{3} \approx 21.333$ achieving a market share of $\frac{1}{11}$ to the duopoly's market share of $\frac{10}{11}$.}

	\item Recall that in (c), we solved $P = 26, Q_d = 160, Q_f = 32$. With these values, and the inverse demand $P = 50 - \frac{1}{5}Q$, we can solve for CS simply with the expression $\frac{1}{2}(50 - 26) (160 + 32)$, which is equal to 2304. Producer surplus is a little more complicated, but can be found simply by solving for the profits of the dominant firm and the fringe, then adding them together. The dominant firm's profits are also relatively simple; $\pi_d = 160 (26 - 10) = 2560$. To solve for the fringe profits, we determine the profits of an individual fringe firm and then multiply by 100: $\pi_f = 100 * (\frac{1}{2} (\frac{32}{100})(26-10)) = 256$. Note that while we subtract 10 in both of these, it's for different reasons; the dominant firm faces a constant cost of 10 so we can directly solve for its profit by plugging in cost, whereas for the fringe we're trying to calculate the area of a triangle on the graph and 10 is the minimum price where it sells. In total, adding these together yields a combined producer surplus of $2560 + 256 = 2816$.
	
	We basically repeat the process again for (d), but using different values. For the purposes of calculating producer surplus, we can treat the two dominant firms as one firm, so $P=\frac{62}{3}, Q_d = \frac{640}{3}, Q_f = \frac{64}{3}$. Plugging these values into the same equations, $CS = \frac{1}{2}(50-\frac{62}{3})(\frac{640 + 64}{3}) = \frac{30976}{3} \approx 3441.778$. Profit for the dominant "firm" (combined two firms) is $\pi_d = \frac{640}{3}(\frac{62}{3}-10) = \frac{20480}{9} \approx 2275.556$ and the total fringe profit is $\pi_f = \frac{1}{2} (\frac{64}{3})(\frac{62}{3}-10) = \frac{1024}{9} \approx 113.778$, giving us a total PS of $\frac{20480}{9}+\frac{1024}{9} \approx 2389.333$.

	To determine deadweight loss for both, we need to solve for the total welfare in the perfectly competitive outcome i.e. when $P=MC$ for the dominant firm (and there are no fringe firms, as they would get outcompeted by the dominant firm's lower costs). In this scenario, $P=10$, meaning $Q=400-80 = 320$. Producers receive no profit/surplus here, but consumer surplus is $CS = \frac{1}{2} (50-10)(320) = 6400$, which also gives us total welfare. Subtracting the total welfare in (c) and (d) gives us the DWL for each, which is equal to to $6400-(2304+2816)=1280$ in part (c) and (approximately) $6400 - (3441.778 + 2389.333) = 568.889$ in part (d). As expected, the dominant duopoly produces less DWL than the dominant monopolist.

	{\color{blue} \textbf{Solution:} In (c), $CS = 2304, PS = 2816, DWL=1280$. In (d), $CS \approx 3441.8, PS \approx 2389.3, DWL \approx 568.9$.}

\end{enumerate}

\item[2.]

\begin{enumerate}
	\item Let $q_1$ and $q_2$ refer to $X$ and $Y$ as defined in the problem, and $Q$ to represent the combined total produced. We then solve for the quantities produced by maximizing firm 1's profit function (which will give us both quantities via symmetry):
	\begin{align*}
		max_{q_1} \pi_1 &= (280 - 2q_1 - 2q_2 - 40)*q_1 \\
		\frac{\partial\pi_1}{\partial q_1} &= 240 - 4q_1 - 2q_2 = 0 \\
		q_1 &= 60 - \frac{1}{2}q_2 \\
		\therefore q_2 &= 60 - \frac{1}{2}q_1 \\
		\therefore q_1 &= 60 - \frac{1}{2}(60 - \frac{1}{2}q_1) \\
		&= 30 + \frac{1}{4}q_1 \\
		\therefore q_1 &= 40 \\
		\therefore q_2 &= 60 - 40/2 = 40
	\end{align*}

	Solving for $P$ and then profits is simply a matter of plugging in the solved quantities: $P=280 - 2(40+40) = 120, \pi_1 = \pi_2 = (120-40)(40) = 3200$.

	{\color{blue} \textbf{Solution:} In Cournot equilibrium, $q_1 = q_2 = 40, P=120, \pi_1=\pi_2 = 3200$.}

	\item From part (a), we already know firm 2's reaction function $q_2 = 60 - \frac{1}{2}q_2$. Only firm 1's reaction function changes here, and we solve for it as such:
	\begin{align*}
		max_{q_1} \pi_1 &= (280 - 2q_1 - 2(60 - \frac{1}{2}q_2) - 40)*q_1 \\
		&= (120-q_1)(q_1) \\
		\frac{\partial\pi_1}{\partial q_1} &= 120 - 2q_1 = 0 \\
		\therefore q_1 &= 60 \\
		\therefore q_2 &= 60 - 60/2 = 30
	\end{align*}

	We then use these quantities to solve for $P$ and profits: $P = 280 - 2(60 + 30) = 100, \pi_1 = (100-40)(60) = 3600, \pi_2 = (100-40)(30) = 1800$

	{\color{blue}\textbf{Solution:} In Stackelberg equilibrium with firm 1 choosing first, $q_1 = 60, q_2 = 30, P = 100, \pi_1 = 3600, \pi_2 = 1800$}
\end{enumerate}


\end{enumerate}
	
\end{document}
