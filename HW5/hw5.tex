\documentclass[12pt,letterpaper]{article}

\usepackage{graphicx,amssymb,amsmath,bm,color,multicol}
\usepackage{../newcommand}
\sloppy
\newcommand{\ignore}[1]{}

\newenvironment{proof}{\noindent{\bf Proof:}}{\qed\bigskip}

\newtheorem{theorem}{Theorem}
\newtheorem{corollary}{Corollary}
\newtheorem{lemma}{Lemma} 
\newtheorem{claim}{Claim}
\newtheorem{fact}{Fact}
\newtheorem{definition}{Definition}
\newtheorem{assumption}{Assumption}
\newtheorem{observation}{Observation}
\newtheorem{example}{Example}
\newcommand{\qed}{\rule{7pt}{7pt}}

\usepackage{times}
\usepackage{ulem}
\usepackage[nocenter]{qtree}
\usepackage{tree-dvips}
\usepackage{gb4e}

\newcommand{\homework}[4]{
	\thispagestyle{plain} 
	\newpage
	\setcounter{page}{1}
	\noindent
	\begin{center}
		\framebox{ \vbox{ \hbox to 6.28in
				{\bf ECON 4190: Industrial Organization \hfill #1} %change course name
				\vspace{4mm}
				\hbox to 6.28in
				{\hspace{2.5in}\large\mbox{Homework #2}}
				\vspace{4mm}
				\hbox to 6.28in
				{{\it Collaborators: #3\hfill}}
			}}
		\end{center}
	}

\oddsidemargin 0in
\evensidemargin 0in
\textwidth 6.5in
\topmargin -0.5in
\textheight 9.0in

\begin{document}

% Modify this command to be your name and computing ID
\homework{Fall 2021}{$5$}{Alex Shen (as5gd), Sean Velhagen (spv5hq), Max Bresticker (mtb9sex)}

Pledge: On my honor, I pledge that I have neither given nor received help on this assignment 
Signature: \textit{Alex Shen, Sean Velhagen, Max Bresticker}

\begin{enumerate}
	
\item[3.2)] 

\item[4.1)]

\item[4.3)] The game tree here can be visualized as follows:

{\color{red}picture here}
                                      
In scenarios where Firm 2 discloses its cost, Firm 1's response is very simple - it will produce the Cournot quantity (and thus receive Cournot equilibrium profits) when $c_2 = 0$, and produce the monpolist quantity (receiving monopolist profits) when $c_2$ is sufficiently high to prevent Firm 2 from entering the market. 

When Firm 2 doesn't disclose its cost, Firm 1 can only maximize its profit function using an expected value for $q_2$. Since the probability of both cases is $p=0.5$, $E(p_2)$ can be determined simply by halving the quantity Firm 2 produces in the Cournot equilibrium. Given this value, we can simply derive Firm 1's best response as follows:

{\color{red} math here}

\item[4)]

\begin{enumerate}

    \item In general, the profit function for a monopoly (colluding firms) is as follows:
    \begin{align*}
        \pi^m &= P(Q)Q - c(Q)
    \end{align*}
    
    In this case, $P(Q) = a-bQ$ and $c(Q) = cQ$, so:
    \begin{align*}
        \pi^m &= (a-bQ)Q - cQ\\
        FOC_{wrt.Q}: 0 &= a-2bQ^m - c\\
        Q^m &= \frac{a-c}{2b}
    \end{align*}
    
    Firm i's profit function in a 2-firm Cournot model with the given inverse demand and cost function is as follows:
    \begin{align*}
        \pi_i &= (a - b(q_1 + q_2))q_i - cq_i
    \end{align*}
    
    So, the firms' profit functions are as follows:
    \begin{align*}
        \pi_1 &= (a - b(q_1 + q_2))q_1 - cq_1\\
        \pi_2 &= (a - b(q_1 + q_2))q_2 - cq_2
    \end{align*}
    
    It is easily seen here that optimizing firm 1's profits will give a similar expression to optimizing firm 2's profits, just with $q_1$ and $q_2$ interchanged. Let us optimize firm 1:
    \begin{align*}
        FOC_{wrt.q_1}: 0 &= -bq_1^c + (a-b(q_1^c + q_2)) - c\\
        0 &= a - 2bq_1^c - bq_2 - c\\
        q_1^c &= \frac{a - bq_2 - c}{2b}
    \end{align*}
    
    We may interchange $q_1$ and $q_2$ to find $q_2^c$:
    \begin{align*}
        q_2^c &= \frac{a - bq_1 - c}{2b}
    \end{align*}
    
    To find the Cournot equilibrium, we can plug in firm 2's best response  function into firm 1's best response function and solve for $q_1^c$ as a function of constants. Again, we see here that the process for solving for $q_1^c$ as a function of constants is similar to the process for solving for $q_2^c$ as a function of constants, just with $q_1^c$ and $q_2^c$ flipped. Plugging $q_2^c$ into firm 1's best response function, we have:
    \begin{align*}
        q_1^c &= \frac{a - b(\frac{a - bq_1^c - c}{2b}) - c}{2b}\\
        q_1^c &= \frac{a}{2b} - \frac{a}{4b} + \frac{q_1^c}{4} + \frac{c}{4b} - \frac{c}{2b}\\
        \frac{3}{4}q_1^c &= \frac{a}{4b} - \frac{c}{4b}\\
        q_1^c &= \frac{a-c}{3b}
    \end{align*}
    
    We may interchange $q_1^c$ and $q_2^c$ to find $q_2^c$ as a function of constants:
    \begin{align*}
        q_2^c &= \frac{a-c}{3b}\\
        Q &= q_1 + q_2\\
        Q^c &= q_1^c + q_2^c\\
        Q^c &= \frac{2(a-c)}{3b}\\
        Q^m &= \frac{a-c}{2b}\\
        \frac{2(a-c)}{3b} &> \frac{a-c}{2b}\\
        Q^c &> Q^m \ \blacksquare
    \end{align*}
    
    \item As previously solved for in part 4.a, firm i's profit under a Cournot system is $\pi_i = (a-b(\frac{2(a-c)}{3b}))(\frac{(a-c)}{3b}) - c(\frac{(a-c)}{3b}) = \frac{1}{9b}(a-c)^2$. For both firms to choose to collude, they would each need profits equal to or exceeding what they would have gotten had they competed under a Cournot system. Given that each firm is colluding, they would collectively produce the monopoly quantity: $Q^m$. If firms cannot freely transfer profits, then each firm must produce a quantity that satisfies their respective profit condition.
    
    \item $MC(q_i) = cq_i$, so the profit function for firm i is as follows:
    \begin{align*}
        \pi_i &= (a - b(q_1 + q_2))q_i - \frac{cq_i^2}{2}
    \end{align*}
    
    We can find $q_1^c$ and $q_2^c$ using the same techniques that we used in 4.a:
    \begin{align*}
        \pi_1 &= (a - b(q_1 + q_2))q_1 - \frac{cq_1^2}{2}\\
        FOC_{wrt.q_1}: 0 &= -bq_1^c + (a-b(q_1^c + q_2)) - cq_1^c\\
        0 &= a - (2b + c)q_1^c - bq_2\\
        q_1^c &= \frac{a - bq_2}{2b + c}
    \end{align*}
    
    By symmetry, we have:
    \begin{align*}
        q_2^c &= \frac{a - bq_1^c}{2b + c}
    \end{align*}
    
    Again, we plug $q_2^c$ into $q_1^c$ to get $q_1^c$ as a function of constants:
    \begin{align*}
        q_1^c &= \frac{a - b(\frac{a - bq_1^c}{2b + c})}{2b + c}\\
        q_1^c &= \frac{a}{2b+c} - \frac{ab}{(2b+c)^2} + \frac{b^2q_1^c}{(2b+c)^2}\\
        (1-\frac{b^2}{(2b+c)^2})q_1^c &= \frac{a(b+c)}{(2b+c)^2}\\
        ((2b+c)^2 - b^2)q_1^c &= a(b+c)\\
        q_1^c &= \frac{a(b+c)}{3b^2 + 4bc + c^2}
    \end{align*}
    
    By symmetry, we have:
    \begin{align*}
        q_2^c &= \frac{a(b+c)}{3b^2 + 4bc + c^2}\\
        Q^c &= q_1^c + q_2^c\\
        \therefore Q^c &= \frac{2a(b+c)}{3b^2 + 4bc + c^2}
    \end{align*}
    
    \item If the firms do not have to worry about getting caught, then the firms should simply have the firm with the lower marginal cost produce all of the quantity and have the lower-cost firm transfer an amount of profit equal to or greater than the other firm's profit under a Cournot system. This is to give the other firm an incentive to collude. If, instead, the firms do have to worry about getting caught (and, therefore, cannot transfer profits), then each firm would have to act like they're still competing in some capacity. Therefore, each firm must produce a quantity such that their respective profits are higher than they would be under a Cournot system, similarly to 4.b.

\end{enumerate}


\end{enumerate}
	
\end{document}
